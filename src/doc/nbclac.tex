\documentclass{revtex4}

\usepackage{graphicx}
\usepackage{amssymb,amsmath}
\usepackage{epstopdf}
\usepackage{bm}
\usepackage{dsfont}
\DeclareGraphicsRule{.tif}{png}{.png}{`convert #1 `dirname #1`/`basename #1 .tif`.png}

\begin{document}
\title{NBC+Lactates}
\maketitle

\section{Fluxes and metabolism}

\subsection{Passive Fluxes}
For each species $X$ that may flow through the membrane, we use the GHK flux equation and find
\begin{equation}
	j_X = - P_X\left(\zeta\right) \Psi(z_X\zeta)
	\left([X]_{out} - [X]e^{z_X\zeta}\right)
\end{equation}
then
\begin{equation}
	\partial_t [X]_{passive} = -\dfrac{S}{V} j_X = \lambda_X.
\end{equation}

\subsection{Electrogenic fluxes}
We the the $NaK$-ATPase
\begin{equation}
	\dfrac{1}{2} \partial_t \left[K^+\right]_{{NaK}} = -\dfrac{1}{3} \partial_t \left[ Na^+\right]_{{NaK}}
	= \dfrac{S}{V}\rho_{NaK}
\end{equation}

The charge conservation imposes the change in potential
\begin{equation}
	\partial_t\left( C E\right) + \sum_X z_X F \partial_t\left( V \times  (\left[X\right]_{out}-\left[X\right]) \right) = 0
\end{equation}
taking care of the units (here concentrations must be in $mol/m^3$.)
We may use
\begin{equation}
	E = ...
\end{equation}

\subsection{Electroneutral Fluxes}
\begin{itemize}
	\item $NHE$, 
	\begin{equation}
	\partial_t \left[Na^+\right]_{NHE} = - \partial_t \left[H^+\right]_{NHE} = \dfrac{S}{V} \rho_{NHE}
	\end{equation}
	
	\item $AE$,
	\begin{equation}
	\partial_t \left[Cl^-\right]_{AE} = -\partial_t \left[HCO_3^-\right]_{AE} = \dfrac{S}{V} \rho_{AE}
	\end{equation}
	
	\item $NBC$,
	\begin{equation}
	\partial_t \left[Na^+\right]_{NBC} = \partial_t \left[[HCO_3^-\right]_{NBC} = \dfrac{S}{V} \rho_{NBC}
	\end{equation}
		
	\item Lactic acid, with $MCT_u$
	\begin{equation}
	\partial_t \left[LacH\right] = \sum_u \dfrac{S}{V}\rho_{MCT_u}
	\end{equation}
\end{itemize}

\subsection{Internal Production}

\subsubsection{$CO_2$}
The $CO_2$ transport is equivalent to an effective $CO_2$ partial pressure $\Pi_{CO_2}$.

\subsubsection{Lactic acid}
We assume that the lactic acid is produced with and internal rate
\begin{equation}
	\Lambda = \Lambda_0 \times \sigma_{PFK}\left(\dfrac{h}{K_{PFK}}\right)
\end{equation}

\section{Chemical Reactions}

\subsection{Water}
\begin{equation}
	H_2O \rightleftharpoons H^+ + HO^-, \; K_w = 10^{-14} = \left[H^+\right]\left[HO^-\right]
\end{equation}	

\subsection{Carbonic system}
\begin{equation}
	CO_2 \rightleftharpoons H^+ + HCO_3^-, \; K_1' = \dfrac{K_1}{K_H} = \dfrac{\left[H^+\right]\left[HCO_3^-\right]}{\Pi_{CO_2}}
\end{equation}
\begin{equation}
	HCO_3^- \rightleftharpoons H^+ + CO_3^{2-}, \; K_2  \dfrac{\left[H^+\right]\left[CO_3^{2-}\right]}{\left[HCO_3^-\right]}
\end{equation}
\subsection{Lactic Acid}
\begin{equation}
	LacH \rightleftharpoons H^+ + Lac^-, \; K_{Lac} = \dfrac{\left[H^+\right]\left[Lac^-\right]}{\left[LacH\right]}
\end{equation}

\section{ODE}
Using $Na^+,K^+$ and $Cl^-$ leaks	
\begin{equation}
\begin{array}{rcl}
	\partial_t E & = & \\
	\partial_t \left[H^+\right] & = \\
	\partial_t \left[HO^-\right] & = \\
\end{array}
\end{equation}


\end{document}

