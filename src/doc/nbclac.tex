\documentclass{revtex4}

\usepackage{graphicx}
\usepackage{amssymb,amsmath}
\usepackage{epstopdf}
\usepackage{bm}
\usepackage{dsfont}
\DeclareGraphicsRule{.tif}{png}{.png}{`convert #1 `dirname #1`/`basename #1 .tif`.png}

\begin{document}
\title{NBC+Lactates}
\maketitle

\section{Fluxes and metabolism}

\subsection{Passive Fluxes}
For each species $X$ that may flow through the membrane, we use the GHK flux equation and find the intake flux
\begin{equation}
	j_X =  P_X\left(\zeta\right) \Psi(z_X\zeta)
	\left([X]_{out} - [X]e^{z_X\zeta}\right)
\end{equation}
then
\begin{equation}
	\partial_t [X]_{passive} = \dfrac{S}{V} j_X = \lambda_X.
\end{equation}
The permeabilities are provided by electrophysiological measurements.

\subsection{Electrogenic fluxes}
We the the $NaK$-ATPase
\begin{equation}
	\dfrac{1}{2} \partial_t \left[K^+\right]_{{NaK}} = -\dfrac{1}{3} \partial_t \left[ Na^+\right]_{{NaK}}
	= \dfrac{S}{V}\rho_{NaK}
\end{equation}

The charge conservation imposes the change in potential
\begin{equation}
	\partial_t\left( C E\right) + \sum_X z_X F \partial_t\left( V \times  (\left[X\right]_{out}-\left[X\right]) \right) = 0
\end{equation}
taking care of the units (here concentrations must be in $mol/m^3$.)

\subsection{Electroneutral Fluxes}
\begin{itemize}
	\item $NHE$, 
	\begin{equation}
	\partial_t \left[Na^+\right]_{NHE} = - \partial_t \left[H^+\right]_{NHE} = \dfrac{S}{V} \rho_{NHE}
	\end{equation}
	
	\item $AE$,
	\begin{equation}
	\partial_t \left[Cl^-\right]_{AE} = -\partial_t \left[HCO_3^-\right]_{AE} = \dfrac{S}{V} \rho_{AE}
	\end{equation}
	
	\item $NBC$,
	\begin{equation}
	\partial_t \left[Na^+\right]_{NBC} = \partial_t \left[HCO_3^-\right]_{NBC} = \dfrac{S}{V} \rho_{NBC}
	\end{equation}
		
	\item Lactic acid, with $MCT_u$
	\begin{equation}
	\partial_t \left[LacH\right] = \sum_u \dfrac{S}{V}\rho_{MCT_u}
	\end{equation}
\end{itemize}

\subsection{Internal Production}

\subsubsection{$CO_2$}
The $CO_2$ transport is equivalent to an effective $CO_2$ partial pressure $\Pi_{CO_2}$.

\subsubsection{Lactic acid}
We assume that the lactic acid is produced with and internal rate
\begin{equation}
	\Lambda = \Lambda_0 \times \sigma_{PFK}\left(\dfrac{h}{K_{PFK}}\right)
\end{equation}

\section{Chemical Reactions}

\subsection{Water}
\begin{equation}
	H_2O \rightleftharpoons H^+ + HO^-, \; K_w = 10^{-14} = \left[H^+\right]\left[HO^-\right]
\end{equation}	

\subsection{Carbonic system}
\begin{equation}
	CO_2 \rightleftharpoons H^+ + HCO_3^-, \; K_1' = \dfrac{K_1}{K_H} = \dfrac{\left[H^+\right]\left[HCO_3^-\right]}{\Pi_{CO_2}}
\end{equation}
\begin{equation}
	HCO_3^- \rightleftharpoons H^+ + CO_3^{2-}, \; K_2  = \dfrac{\left[H^+\right]\left[CO_3^{2-}\right]}{\left[HCO_3^-\right]}
\end{equation}
\subsection{Lactic Acid}
\begin{equation}
	LacH \rightleftharpoons H^+ + Lac^-, \; K_{Lac} = \dfrac{\left[H^+\right]\left[Lac^-\right]}{\left[LacH\right]}
\end{equation}

\section{ODE}
\subsection{Matter transformation}
Using $Na^+,K^+$ and $Cl^-$ leaks, and we take into account the molar rate of each reaction.	
\begin{equation}
\left\lbrace
\begin{array}{rcl}
	\partial_t \left(C_m S E \right) & = & FS\left(j_{Na^+}+j_{K^+}-j_{Cl^-} - \rho_{NaK}\right) \\
	\\
	\partial_t \left[Na^+\right] & = & \dfrac{S}{V}\left(j_{Na^+}+\rho_{NHE}+\rho_{NBC}-3\rho_{NaK}\right) \\
	\\
	\partial_t \left[Cl^-\right] & = & \dfrac{S}{V}\left(j_{Cl^-} + \rho_{AE}\right)\\
	\\
	\partial_t \left[K^+\right]  & = & \dfrac{S}{V}\left(j_{K^+}+2\rho_{NaK}\right)\\
	\\
	\partial_t \left[H^+\right]  & = & \dfrac{S}{V}\left(-\rho_{NHE}\right) + \chi_{w} + \chi_{1'} + \chi_2 + \chi_{Lac}\\
	\\
	\partial_t \left[HO^-\right] & = & \chi_{w}\\
	\\
	\partial_t \left[HCO_3^-\right] & = & \dfrac{S}{V}\left(\rho_{NBC}-\rho_{AE}\right) + \chi_{1'} - \chi_2 \\
	\\
	\partial_t \left[CO_3^{2-}\right] & = & \chi_2\\
	\\
	\partial_t \left[LacH\right]      & = & \Lambda - \dfrac{S}{V}\left(\sum_u\rho_{MCT_u}\right) + \chi_{Lac}\\
	\\
	\partial_t \left[Lac^-\right]     & = & - \chi_{Lac}\\
	
\end{array}
\right.
\end{equation}

\subsection{Steady State}
\begin{itemize}
\item
The potential is defined by
\begin{equation}
	0 = \dfrac{3}{2}j_{K^+}^\star + j_{Na^+}^\star - j_{Cl^-}^\star
\end{equation}
and will produce the GHK potential equation with a regulation by $NaK$-ATPase.
There is one potential for a given set of permeabilities and inside/outside concentration.
So the permeabilities may be adjusted to set $E_m^\star$ at will.
\item One the fluxes are computed, then $V_{NaK}$ may be computed using $K^+$ intake by
	\begin{equation}
		0 = j_{K^+}^\star + 2 \rho_{NaK}^\star.
	\end{equation}
\item Likewise, $V_{AE}$ may be computed using the $Cl^-$ intake by
	\begin{equation}
		0 = j_{Cl^-}^\star + \rho_{AE}^\star
	\end{equation}
	The flux of chloride must be negative, since $AE$ should work as a chloride intake.
\item Finally, we obtain
\begin{equation}
	0 = \rho_{NBC}^\star+\rho_{NHE}^\star - \rho_{AE}^\star.
\end{equation}
Since $NBC$ and $NHE$ are chemically redundant, we must introduce another parameter $\alpha\in[0:1]$
such that 
\begin{equation}
	\left\lbrace
	\begin{array}{rcr}
	\rho_{NHE}^\star & = & \alpha \rho_{AE}^\star\\
	\rho_{NBC}^\star & = & (1-\alpha) \rho_{AE}^\star\\ 
	\end{array}
	\right.
\end{equation}


\item Since we got an open system in lactic acid, we need a basal level of lactic acid (or lactate) concentration.
Assuming that the production term $\Lambda_0$ is known as well, then we find the expression
\begin{equation}
	\Lambda_0 \sigma_{PFK}^\star = \Lambda^\star = \rho_{MCT_1} + \rho_{MCT_4}.
\end{equation}
We also need a parameter $\beta$ such that
\begin{equation}
	\left\lbrace
	\begin{array}{rcr}
	\rho_{MCT_1} & = & \beta \Lambda^\star\\
	\rho_{MCT_4} & = & (1-\beta) \Lambda^\star\\
	\end{array}
	\right.
\end{equation}

\end{itemize}

\subsection{System Reduction}
We use the protic vector of 6 variables
\begin{equation}
	\vec{X} = \left( \left[H^+\right]  \left[HO^-\right] \left[HCO_3^-\right] \left[CO_3^{2-}\right] \left[LacH\right]  \left[Lac^-\right] \right)^T
\end{equation}
undergoing 4 reactions.
The topological matrix is
\begin{equation}
	\boldsymbol{\nu} =
	\begin{pmatrix}
	1 & 1 & 0 & 0 & 0 & 0\\
	1 & 0 & 1 & 0 & 0 & 0\\
	1 & 0 & -1 & 1 & 0 & 0\\
	1 & 0 & 0 & 0  & -1 & 1\\
	\end{pmatrix}
\end{equation}
The equilibrium vector is
\begin{equation}
	\vec{\Gamma} =
	 \begin{pmatrix}
	K_w - \left[H^+\right]  \left[HO^-\right]\\
	K_1'\Pi_{CO_2} - \left[H^+\right]\left[HCO_3^-\right]\\
	K_2 {\left[HCO_3^-\right]} -\left[H^+\right]\left[CO_3^{2-}\right]\\
	K_{Lac}\left[LacH\right] - \left[H^+\right]\left[Lac^-\right]
	\end{pmatrix}
\end{equation}

Neglecting the dissociation of $HCO_3^-$, we get the already intricate expression
\begin{equation}
\partial_t h = - \dfrac{S}{V} h^2
\dfrac{\left(h+K_{Lac}\right)\left(\rho_{NHE}+\rho_{NBC}-\rho_{AE}\right)+K_{Lac}\left(\sum \rho_{MCT}-V\Lambda/S\right)}
{h^3+(h+K_{Lac})\left(\kappa+K_w\right)+h K_{Lac}\left(h+\left[LacH\right]\right)}
\end{equation}
We check that if $\left[LacH\right]\equiv0$, then for any value of $K_{Lac}$, the expression is
$$
	\partial_t h\vert_{\left[LacH\right]\equiv0} = 
	-\dfrac{S}{V} h^2
	\dfrac{\left(\rho_{NHE}+\rho_{NBC}-\rho_{AE}\right)}{h^2+K_w+\kappa}
$$

\end{document}

