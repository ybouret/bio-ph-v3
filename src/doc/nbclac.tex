\documentclass{revtex4}

\usepackage{graphicx}
\usepackage{amssymb,amsmath}
\usepackage{epstopdf}
\usepackage{bm}
\usepackage{dsfont}
\DeclareGraphicsRule{.tif}{png}{.png}{`convert #1 `dirname #1`/`basename #1 .tif`.png}

\begin{document}
\title{Establishing a pH equation}
\maketitle

\section{What do we need}
We use the method described in [PLOS one...]. We first inventory the species which are required
to design a minimal biochemical and physical subsystem that will allow us to carry out the computation
of the global ionic steady state of a cell, including the membrane potential.
Those species are naturally $H^+$, $HO^-$, $K^+$, $Na^+$, $Cl^-$, aqueous $CO_2$, $HCO_3^-$, $CO_3^{2-}$ and
for this study the lactic acid $LacH$ and the lactacte ion $Lac^-$.
In the following, we review the  matter fluxes through the cell membrane and the other
chemical transformations that allows the build up of a globally self-consistent physical model.

\section{Fluxes through the cell membrane}


\subsection{Passive Fluxes}
We assume that the internal potential of the cell is $E$, and we define :
\begin{equation}
	\zeta = \dfrac{FE}{RT}.
\end{equation}

For each species $X$ (with an algebraic charge $z_X$) which that may flow through the membrane, we use the Goldman-Hodgkin-Katz flux equation and find the intake flux :
\begin{equation}
	j_X =  P_X\left(\zeta\right) \Psi(z_X\zeta)
	\left([X]_{out} - [X]e^{z_X\zeta}\right), \; \Psi(u) = \dfrac{u}{e^u-1}.
\end{equation}
We then get their corresponding contribution to the change of concentration in a cell with a volume $V$ and
a surface $S$ by:
\begin{equation}
	\partial_t [X]_{passive} = \dfrac{S}{V} j_X = \lambda_X.
\end{equation}
The permeabilities are provided by electrophysiological measurements (see PLOS).
If the charge of $X$ is not zero, then a passive leak is electrogenic.

\subsection{Electrogenic fluxes}
We use the most ubiquitous $NaK$-ATPase which transports the following ions as:
\begin{equation}
	\dfrac{1}{2} \partial_t \left[K^+\right]_{{NaK}} = -\dfrac{1}{3} \partial_t \left[ Na^+\right]_{{NaK}}
	= \dfrac{S}{V}\rho_{NaK}
\end{equation}
with a flux approximated by [PLOS]
\begin{equation}
	\rho_{NaK} = V_{NaK} \dfrac{\left[1+\tanh\left(0.39\zeta+1.28\right)\right]}{2}\dfrac{\left[Na^+\right]}{K_{NaK}+\left[Na^+\right]},\;K_{NaK}\simeq12mM.
\end{equation}

The charge conservation imposes the change in potential
\begin{equation}
	\partial_t\left( C E\right) + \sum_X z_X F \partial_t\left( V \times  (\left[X\right]_{out}-\left[X\right]) \right) = 0
\end{equation}
We take care of the units in which this equation is valid (here concentrations must be in $mol/m^3$.)

\subsection{Electroneutral Fluxes}
The following transporters are present in the cell we described, and their kinetics are well described.
\begin{itemize}
	\item $NHE$, 
	\begin{equation}
	\partial_t \left[Na^+\right]_{NHE} = - \partial_t \left[H^+\right]_{NHE} = \dfrac{S}{V} \rho_{NHE}
	\end{equation}
	with
	\begin{equation}
		\rho_{NHE} = V_{NHE} \sigma_{NHE}\left(\dfrac{h}{K_r}\right), \; h=\left[H^+\right], \;K_r\approx1.8\cdot10^{-8}M
	\end{equation}
	and
	\begin{equation}
		\sigma_{NHE}(x) = \dfrac{
		x(1+x) + L_0 C x(1+Cx)
		}
		{
		L_0 (1+Cx)^2 + (1+x)^2
		}, \; K_t\approx 3.6\cdot10^{-6}M, \; C=K_r/K_t,\;L_0\approx10^3.
	\end{equation}
	\item $AE$,
	\begin{equation}
	\partial_t \left[Cl^-\right]_{AE} = -\partial_t \left[HCO_3^-\right]_{AE} = \dfrac{S}{V} \rho_{AE}
	\end{equation}
	with
	\begin{equation}
	\rho_{AE} = V_{AE} \dfrac{\left[HCO_3^-\right]}{K_{AE} + \left[HCO_3^-\right]},\;\; K_{AE}\simeq10mM.
	\end{equation}
	
	\item $NBC$,
	\begin{equation}
	\partial_t \left[Na^+\right]_{NBC} = \partial_t \left[HCO_3^-\right]_{NBC} = \dfrac{S}{V} \rho_{NBC}.
	\end{equation}
	In a first approximation, $NBC$ works as a two-components, reversible cotransporter
	\begin{equation}
		\rho_{NBC} = V_{NBC}
		 \left[ 
		 	\sigma_{NBC}\left(\left[Na^+\right]_{out},\left[HCO_3^-\right]_{out}\right) 
		  - \sigma_{NBC}\left(\left[Na^+\right],\left[HCO_3^-\right]\right) 
		 \right]
	\end{equation}
	and
	\begin{equation}
	\sigma_{NBC}\left(\left[Na^+\right],\left[HCO_3^-\right]\right) =
	\dfrac{\left[Na^+\right]\left[HCO_3^-\right]}
	{
		\left[Na^+\right]\left[HCO_3^-\right]+\dfrac{\left[Na^+\right]}{K_{Na}}+\dfrac{\left[HCO_3^-\right]}{K_{BC}},
	}, \;\; K_{Na} \simeq 30 mM, \; K_{BC} \simeq  4mM.
	\end{equation}	
		
	\item Lactic acid (or lactate+proton) is expelled through Mono-Carboxylate Transporters (MCT),
	mostly $MCT_1$ and $MCT_4$, so that
	\begin{equation}
	\partial_t \left[LacH\right]_{MCT} = \sum_u \dfrac{S}{V}\rho_{MCT_u}.
	\end{equation}
	The kinetics expression of this term is expresses further, according to the hypotheses of our model.
	
\end{itemize}

\subsection{Internal Production}

\subsubsection{$CO_2$}
The aqueous $CO_2$ presence is equivalent to an effective $CO_2$ partial pressure $\Pi_{CO_2}\approx 40mmHg$,
arising from the normal metabolism.

\subsubsection{Lactic acid}
We assume that the lactic acid is produced with and internal rate $\Lambda$, which may be
intrinsic or simulated to see how the cell reacts to different lactic concentrations.

From another article [Bock and Frieden], we may monitor the activity of PFK since it has
been shown that in our case it may be described by:
\begin{equation}
\sigma_{PFK}\left(\mathrm{pH}\right)  = \left(\dfrac{1+\tanh\left[ a \left(\mathrm{pH}-b\right)\right] }{2}\right)^2
\end{equation}
with
\begin{equation}
	\left\lbrace
	\begin{array}{rcll}
	a & = & 4.18 & \pm 0.04\\
	b & = & 6.81 & \pm 0.003\\
	\end{array}
	\right.
\end{equation}


\section{Chemical Equilibria}
The following equilibria are to be accounted for. Each of them possess a molar rate that we will use
in the next paragraph.
\begin{itemize}
\item{Water}
\begin{equation}
	H_2O \rightleftharpoons H^+ + HO^-, \; K_w = 10^{-14} = \left[H^+\right]\left[HO^-\right]
\end{equation}	
with a molar rate $\chi_w$.

\item{Carbonic system}
\begin{equation}
	CO_2 \rightleftharpoons H^+ + HCO_3^-, \; K_1' = \dfrac{K_1}{K_H} = \dfrac{\left[H^+\right]\left[HCO_3^-\right]}{\Pi_{CO_2}}
	, \; K_1=4.45\cdot10^{-7},\;K_H=29.41 atm/M.
\end{equation}
with a molar rate $\chi_1'$ and

\begin{equation}
	HCO_3^- \rightleftharpoons H^+ + CO_3^{2-}, \; K_2  = \dfrac{\left[H^+\right]\left[CO_3^{2-}\right]}{\left[HCO_3^-\right]}
	,\;K_2 = 5.6\cdot10^{-11}.
\end{equation}
with a molar rate $\chi_2$.

\item{Lactic Acid}
\begin{equation}
	LacH \rightleftharpoons H^+ + Lac^-, \; K_{Lac} = \dfrac{\left[H^+\right]\left[Lac^-\right]}{\left[LacH\right]},
	\;K_{Lac} = 10^{-3.86}.
\end{equation}
with a molar rate $\chi_{Lac}$.
\end{itemize}


\section{Differential system}
\subsection{Matter transformation}
Using only $Na^+,K^+$ and $Cl^-$ leaks, the net molar rate for each species is
listed below.	
\begin{equation}
\left\lbrace
\begin{array}{rcl}
	\partial_t \left(C_m S E \right) & = & FS\left(j_{Na^+}+j_{K^+}-j_{Cl^-} - \rho_{NaK}\right) \\
	\\
	\partial_t \left[Na^+\right] & = & \dfrac{S}{V}\left(j_{Na^+}+\rho_{NHE}+\rho_{NBC}-3\rho_{NaK}\right) \\
	\\
	\partial_t \left[Cl^-\right] & = & \dfrac{S}{V}\left(j_{Cl^-} + \rho_{AE}\right)\\
	\\
	\partial_t \left[K^+\right]  & = & \dfrac{S}{V}\left(j_{K^+}+2\rho_{NaK}\right)\\
	\\
	\partial_t \left[H^+\right]  & = & \dfrac{S}{V}\left(-\rho_{NHE}\right) + \chi_{w} + \chi_{1'} + \chi_2 + \chi_{Lac}\\
	\\
	\partial_t \left[HO^-\right] & = & \chi_{w}\\
	\\
	\partial_t \left[HCO_3^-\right] & = & \dfrac{S}{V}\left(\rho_{NBC}-\rho_{AE}\right) + \chi_{1'} - \chi_2 \\
	\\
	\partial_t \left[CO_3^{2-}\right] & = & \chi_2\\
	\\
	\partial_t \left[LacH\right]      & = & \Lambda - \dfrac{S}{V}\left(\sum_u\rho_{MCT_u}\right) + \chi_{Lac}\\
	\\
	\partial_t \left[Lac^-\right]     & = & - \chi_{Lac}\\
	
\end{array}
\right.
\end{equation}

\subsection{Finding out the Steady State}
\begin{itemize}
\item We assume that $h^\star$ is known.
\item
The potential is defined by
\begin{equation}
	0 = \dfrac{3}{2}j_{K^+}^\star + j_{Na^+}^\star - j_{Cl^-}^\star
\end{equation}
and will produce the GHK potential equation with a regulation by $NaK$-ATPase.
There is one potential for a given set of permeabilities and inside/outside concentration.
So the permeabilities may be adjusted to set $E_m^\star$ at will.
\item One the fluxes are computed, then $V_{NaK}$ may be computed using $K^+$ intake by
	\begin{equation}
		0 = j_{K^+}^\star + 2 \rho_{NaK}^\star.
	\end{equation}
\item Likewise, $V_{AE}$ may be computed using the $Cl^-$ intake by
	\begin{equation}
		0 = j_{Cl^-}^\star + \rho_{AE}^\star
	\end{equation}
	The flux of chloride must be negative, since $AE$ should work as a chloride intake.
\item Finally, we obtain
\begin{equation}
	0 = \rho_{NBC}^\star+\rho_{NHE}^\star - \rho_{AE}^\star.
\end{equation}
Since $NBC$ and $NHE$ are chemically redundant, we must introduce another parameter $\alpha\in[0:1]$
such that 
\begin{equation}
	\left\lbrace
	\begin{array}{rcr}
	\rho_{NHE}^\star & = & \alpha \rho_{AE}^\star\\
	\rho_{NBC}^\star & = & (1-\alpha) \rho_{AE}^\star\\ 
	\end{array}
	\right.
\end{equation}

\item The lactic species concentration will be determined by the struggle between the production
by $\Lambda$ and the elimination by the MCTs. At first, we may consider that we start with no lactic species.


\end{itemize}

And we can add some other equation (like buffering by another weak acid) at will.

\subsection{Findind the chemical rates}
Besides water, it is almost impossible to find a complete kinetic description of all the protic equilibria, since
the proton is involved in a combinatoric number of transformation. Nonetheless, the protic
reactions in water have a very short relaxation time, around a few microseconds, which is very small compared to
the biochemical times we are exploring (way greater than one second).\\
The main idea we previously developped [PLOS] was the following: using a vector of concentration $\vec{X}$, let us assume that we
have a vector $\vec{\Gamma}\left(\vec{X}\right)$ (with a size equal to the number of equilibria) which is nul when all equilibria are met. We now make a 'slow' perturbation $\partial_t \vec{X}_{slow}$ during $dt$, and we look for the chemical extent $d\vec{\xi}$
produced by the equilibria such that
\begin{equation}
		\vec{\Gamma}\left(\vec{X}+dt\partial_t\vec{X}_{slow} + \boldsymbol{\nu}^T d\vec{\xi}\right) = \vec{0}
\end{equation}
where $\boldsymbol{\nu}$ is the topological matrix of the equilibria.
We do not need any more to compute the molar rate, since we obtain a new set of equations:
\begin{equation}
	\partial_t\vec{X} = \partial_t\vec{X}_{slow} - \boldsymbol{\nu}^T\left(\boldsymbol{\Phi}\boldsymbol{\nu}^T\right)^{-1}
	\boldsymbol{\Phi} \partial_t\vec{X}_{slow}
\end{equation}
where $\boldsymbol{\Phi}$ is the Jacobian matrix of $\vec{\Gamma}$.\\
The 'response' matrix has no simple expression, but can be numerically efficiently computed for any simulation.

\subsection{Rates for MCT}
Since in our model all the protic concentrations are dependent, we may model the MCTs fluxes by
\begin{equation}
	\left\lbrace
	\begin{array}{rcll}
	\rho_{MCT_1} & = & V_{m_1} \dfrac{\left[Lac^-\right]}{K_{m_1}+\left[Lac^-\right]}, & K_{m_1}\simeq 5mM\\
	\\
	\rho_{MCT_4} & = & V_{m_4} \dfrac{\left[Lac^-\right]}{K_{m_4}+\left[Lac^-\right]}, & K_{m_4}\simeq 30mM\\
	\end{array}
	\right.
\end{equation}
Approximately, $MCT$s are able to expel at most the equivalent of $\sigma_{Lac}\approx 1mM/min$, so that, assuming the saturation
\begin{equation}
\sigma_{Lac} = \dfrac{S}{V}\left( V_{m_1} + V_{m_4} \right)
\end{equation}
We hence need a parameter $\beta$ such that
\begin{equation}
	\left\lbrace
	\begin{array}{rcr}
	V_{m_1} & = & \beta \dfrac{V}{S} \sigma_{Lac} \\
	\\
	V_{m_4} & = & (1-\beta) \dfrac{V}{S} \sigma_{Lac} \\
	\end{array}
	\right.
\end{equation}
and we plug those expressions in our model.

\subsection{Example of system reduction}
For the sake of simplicity, we get rid of the dissociation of bicarbonate.
We hence use the protic vector of 5 variables
\begin{equation}
	\vec{X} = \left( \left[H^+\right]  \left[HO^-\right] \left[HCO_3^-\right] \left[LacH\right]  \left[Lac^-\right] \right)^T
\end{equation}
undergoing 3 reactions, leading to an equilibrium vector
\begin{equation}
	\vec{\Gamma} =
	 \begin{pmatrix}
	K_w - \left[H^+\right]  \left[HO^-\right]\\
	 \kappa - \left[H^+\right]\left[HCO_3^-\right]\\
	K_{Lac}\left[LacH\right] - \left[H^+\right]\left[Lac^-\right]
	\end{pmatrix}, \;\kappa=K_1'\Pi_{CO_2}
\end{equation}
which is free of singularity.\\
The topological matrix is a $5\times3$ matrix
\begin{equation}
	\boldsymbol{\nu} =
	\begin{pmatrix}
	1 & 1 & 0 &  0 & 0\\
	1 & 0 & 1 &  0 & 0\\
	1 & 0 & 0 &  -1 & 1\\
	\end{pmatrix}.
\end{equation}

In that simple case, the algebraic computation provides the following expression for $\left[H^+\right]$.
\begin{equation}
\partial_t h = - \dfrac{S}{V} h^2
\dfrac{\left(h+K_{Lac}\right)\left(\rho_{NHE}+\rho_{NBC}-\rho_{AE}\right)+K_{Lac}\left(\sum \rho_{MCT}-V\Lambda/S\right)}
{h^3+(h+K_{Lac})\left(\kappa+K_w\right)+h K_{Lac}\left(h+\left[LacH\right]\right)}.
\end{equation}
We check that if $\left[LacH\right]\equiv0$ (ie we remove all the lactic species from the system), then for any value of $K_{Lac}$, the expression is
$$
	\partial_t h\vert_{\left[LacH\right]\equiv0} = 
	-\dfrac{S}{V} h^2
	\dfrac{\left(\rho_{NHE}+\rho_{NBC}-\rho_{AE}\right)}{h^2+K_w+\kappa}
$$

\end{document}

